\documentclass[a4paper,twoside]{article}
\usepackage{fancyhdr}
\usepackage{amssymb,amsmath}
\usepackage{subcaption} % floats in floats
\pagestyle{fancy}
\lhead{}
\chead{}
\rhead{\bfseries Genetic Programming 2018}
\lfoot{Worik Turei Stanton}

\renewcommand{\headrulewidth}{0.4pt}
\renewcommand{\footrulewidth}{0.4pt}
\renewcommand{\baselinestretch}{1.5} % one and a half spacing
\usepackage{graphicx}
\usepackage{url}

\title{Genetic Programming}
\author{Worik Turei Stanton}
\begin{document}

\begin{enumerate}
  \item Record seconds per generation for each simulation
\section{Introduction}

Answer some questions about Genetic Programming.

\begin{enumerate}
\item How does population size relate to performance?
\item How much need is there for mutation?
\item What sorts of problems can be solved?
\end{enumerate}

\section{Abalone}

\url{https://archive.ics.uci.edu/ml/datasets/abalone}

\begin{quote}
  Predicting the age of abalone from physical measurements. The age of
  abalone is determined by cutting the shell through the cone, staining
  it, and counting the number of rings through a microscope -- a boring
  and time-consuming task. Other measurements, which are easier to
  obtain, are used to predict the age....
\end{quote}

\begin{quote}
  Abalone (via Spanish abul\'{o}n, from Rumsen aul\'{o}n) is a common name for
  any of a group of small to very large sea snails, marine gastropod
  molluscs in the family Haliotidae. 
\end{quote}
\url{https://en.wikipedia.org/wiki/Abalone}

\subsection{Attributes}

\begin{table}[h]
  \begin{tabular}{llll}
    Name & Data Type & Meas. & Description\\
    \hline
    Sex &  nominal & &  M, F, and I (infant)\\
    Length & continuous & mm &  Longest shell measurement\\
    Diameter & continuous & mm &  perpendicular to length\\
    Height & continuous & mm &  with meat in shell\\
    Whole weight & continuous & grams & whole abalone\\
    Shucked weight & continuous & grams & weight of meat\\
    Viscera weight & continuous & grams & gut weight (after bleeding)\\
    Shell weight & continuous & grams & after being dried\\
    Rings & integer && +1.5 gives the age in years\\
  \end{tabular}
  \label{tab:abalone.parameters}
  \caption{The attributes.  The last one, \emph{Rings}, is the
    objective to predict }
\end{table}

\subsubsection{Sex}

The sex of the fish is a nominal value in $\{M, F, I\}$.  Given
that the Genetic Programming system uses floating point data the
sex will have to be either ignored or re-coded.

Here two approaches will be used:
\begin{enumerate}
\item As a factor.  Three mutually exclusive flags with one of
  three variables set to $1$ and the other two to $0$
\item Encode sex as $F\Rightarrow-1,\ I\Rightarrow0, M\Rightarrow1$
\end{enumerate}


\subsubsection{Objective}

The objective is the age of the fish.  In some
studies\footnote{E.g., see
  \url{http://citeseerx.ist.psu.edu/viewdoc/download?doi=10.1.1.14.2321&rep=rep1&type=pdf}}
the age is divided into three classes making this a classification
problem.

\begin{tabular}[h]{|ll|}
  \hline
  Rings  &                      Age\\
  0-8    &                    Young\\
  9-14   &                    Adult\\
  15-29  &                    Old \\
  \hline
\end{tabular}

Encode as

\begin{tabular}[h]{lcl}
  -1 &$\Rightarrow$& Young\\
  0 & $\Rightarrow$ & Adult\\
  1 &  $\Rightarrow$  & Old\\
\end{tabular}


Here both approaches will be used


\subsection{Experimental Setup}

Given two encodings of sex, and two of the objective function there
will be four simulation runs.

\begin{description}
\item[Abalone1C] Continuous objective, sex as factor.  See table
  \ref{tab:sample.data.1C} for a sample
\item[Abalone2C]  Objective as classes, sex as factor.  See table
  \ref{tab:sample.data.2C} for a sample
\item[Abalone1N] Continuous objective, sex encoded.  See table
  \ref{tab:sample.data.1N} for a sample
\item[Abalone2N]  Objective as classes, sex encoded.  See table
  \ref{tab:sample.data.2N} for a sample
  
\end{description}

\begin{table}
  \begin{tabular}{lllllllllll}
    F & I & M & Length & Diameter & Height & W.Weight & S.Weight & V.Weight & Sh.Weight & Age \\
    0 & 0 & 1 & 0.455 & 0.365 & 0.095 & 0.514 & 0.2245 & 0.101 & 0.15 & 15 \\
    0 & 0 & 1 & 0.35 & 0.265 & 0.09 & 0.2255 & 0.0995 & 0.0485 & 0.07 & 7 \\
    1 & 0 & 0 & 0.53 & 0.42 & 0.135 & 0.677 & 0.2565 & 0.1415 & 0.21 & 9 \\
  \end{tabular}
  \label{tab:sample.data.1C}
  \caption{Sample data for \textbf{Abalone1C}}
\end{table}


\begin{table}
  \begin{tabular}{lllllllllll}
    F & I & M & Length & Diameter & Height & W.Weight & S.Weight & V.Weight & Sh.Weight & O2 \\
    0 & 0 & 1 & 0.455 & 0.365 & 0.095 & 0.514 & 0.2245 & 0.101 & 0.15 & 1 \\
    0 & 0 & 1 & 0.35 & 0.265 & 0.09 & 0.2255 & 0.0995 & 0.0485 & 0.07 & -1 \\
    1 & 0 & 0 & 0.53 & 0.42 & 0.135 & 0.677 & 0.2565 & 0.1415 & 0.21 & 0 \\
  \end{tabular}
  \label{tab:sample.data.2N}
  \caption{Sample data for \textbf{Abalone2N}}
\end{table}

\begin{table}
  \begin{tabular}{lllllllll}
    Sex & Length & Diameter & Height & W.Weight & S.Weight & V.Weight & Sh.Weight & O1 \\
    1 & 0.455 & 0.365 & 0.095 & 0.514 & 0.2245 & 0.101 & 0.15 & 15 \\
    1 & 0.35 & 0.265 & 0.09 & 0.2255 & 0.0995 & 0.0485 & 0.07 & 7 \\
    -1 & 0.53 & 0.42 & 0.135 & 0.677 & 0.2565 & 0.1415 & 0.21 & 9 \\
  \end{tabular}
  \label{tab:sample.data.1N}
  \caption{Sample data for \textbf{Abalone1N}}
\end{table}

\begin{table}
  \begin{tabular}{lllllllll}
    Sex.1 & Length & Diameter & Height & W.Weight & S.Weight & V.Weight & Sh.Weight & O2 \\
    1 & 0.455 & 0.365 & 0.095 & 0.514 & 0.2245 & 0.101 & 0.15 & 1 \\
    1 & 0.35 & 0.265 & 0.09 & 0.2255 & 0.0995 & 0.0485 & 0.07 & -1 \\
    -1 & 0.53 & 0.42 & 0.135 & 0.677 & 0.2565 & 0.1415 & 0.21 & 0 \\
  \end{tabular}
  \label{tab:sample.data.2C}
  \caption{Sample data for \textbf{Abalone2C}}
\end{table}

\subsubsection{Common Setup}

For this simulation all 

\begin{verbatim}
num_generations 2000
initial_population 1000
max_population 10000
crossover_percent 50
training_percent 80
plot_xlab Age
data_file data2N.in
model_data_file Abalone2N.txt
plot_file Abalone2N.png
r_script_file Abalone2N.R
generations_file AbaloneGenerations2N.txt
birthsanddeaths_file AbaloneBirthsAndDeaths2N.txt
\end{verbatim}

The first six parameters are constant over the four simulations.  The
final six will be specific to each individual simulation

\subsection{Results}

Each simulation took between 116 and 138 minutes real time\footnote{On
  a Intel NUC: Intel Core i3-3217U CPU @ 1.80GHz with 15G RAM}.

None of the simulations produced a model that did a very good job
estimating the age of the fish.

In figure \ref{fig evaluation abalone 1} it can be seen that the
evaluation of the best models in the simulation very quickly reached a
maxima.  Simulations 1C and 1N reached a higher evaluation plateau
than 2C and 2N.  1N and 2N both did a little bit better than 1C and 2C
but the results are close in that case.

A tentative observation is that the continuous objective is easier for
the Genetic Programming simulation to find than a classification.
There is less evidence that encoding the Sex parameter in $-1..1$ is
better than using factors.


Results archived as Abalone20180501

\begin{figure}
  \centering
  \begin{subfigure}[b]{0.4\textwidth}
    \begin{center}
      \includegraphics[height=0.4\textheight]{Fragments/Abalone1C.png}
    \end{center}
    \caption{Results of simulation for Abalone1C}
    \label{fig:results.Abalone1C}
  \end{subfigure}  
\hfill
  \begin{subfigure}[b]{0.4\textwidth}
    \begin{center}
      \includegraphics[height=0.4\textheight]{Fragments/Abalone2C.png}
    \end{center}
    \caption{Results of simulation for Abalone2C}
    \label{fig:results.Abalone2C}
  \end{subfigure}

  \begin{subfigure}[b]{0.4\textwidth}
    \begin{center}
      \includegraphics[height=0.4\textheight]{Fragments/Abalone1N.png}
    \end{center}
    \caption{Results of simulation for Abalone1N}
    \label{fig:results.Abalone1N}
  \end{subfigure}  
\hfill
  \begin{subfigure}[b]{0.4\textwidth}
    \begin{center}
      \includegraphics[height=0.4\textheight]{Fragments/Abalone2N.png}
    \end{center}
    \caption{Results of simulation for Abalone2N}
    \label{fig:results.Abalone2N}
  \end{subfigure}
  \caption{Results of Abalone Age Estimation Simulation}
\end{figure}

\begin{figure}
  \centering
  \begin{subfigure}[b]{0.2\textwidth}
    \begin{center}
      \includegraphics[height=0.2\textheight]{Fragments/Abalone1CGeneration.png}
    \end{center}
    \caption{ Abalone1C}
    \label{fig:evaluation.Abalone1C}
  \end{subfigure}  
~
  \begin{subfigure}[b]{0.2\textwidth}
    \begin{center}
      \includegraphics[height=0.2\textheight]{Fragments/Abalone2CGeneration.png}
    \end{center}
    \caption{ Abalone2C}
    \label{fig:evaluation.Abalone2C}
  \end{subfigure}
~
  \begin{subfigure}[b]{0.2\textwidth}
    \begin{center}
      \includegraphics[height=0.2\textheight]{Fragments/Abalone1NGeneration.png}
    \end{center}
    \caption{ Abalone1N}
    \label{fig:evaluation.Abalone1N}
  \end{subfigure}  
~
  \begin{subfigure}[b]{0.2\textwidth}
    \begin{center}
      \includegraphics[height=0.2\textheight]{Fragments/Abalone2NGeneration.png}
    \end{center}
    \caption{ Abalone2N}
    \label{fig:evaluation.Abalone2N}
  \end{subfigure}
  \label{fig evaluation abalone 1}
  \caption{Evolution of Evaluation for Abalone Age Estimation Simulation}
\end{figure}




\subsection{Increase Population}

Hypothesising that increasing the population size in the simulations
will improve the results the population size increased by a factor of
10.

\begin{verbatim}
initial_population 1000
max_population 10000
\end{verbatim}

\end{document}
