\documentclass[a4paper,twoside]{article}
\usepackage{fancyhdr}
\usepackage{amssymb,amsmath}
\usepackage{subcaption} % floats in floats
\usepackage{listings} % Code listings
\usepackage{amsmath} % For writing step functions
\pagestyle{fancy}
\lhead{}
\chead{}
\rhead{\bfseries Genetic Programming 2018}
\lfoot{Worik Turei Stanton}

\renewcommand{\headrulewidth}{0.4pt}
\renewcommand{\footrulewidth}{0.4pt}
\renewcommand{\baselinestretch}{1.5} % one and a half spacing
\usepackage{graphicx}
\usepackage{url}

\title{Genetic Programming}
\author{Worik Turei Stanton}
\begin{document}

\begin{enumerate}
  \item Record seconds per generation for each simulation
\end{enumerate}

\section{Introduction}

Use Genetic Programming as a classifier



\section{Purpose}
Answer some questions about using Genetic Programming as a classifier.

\begin{enumerate}
\item How does population size relate to performance?
\item How much need is there for mutation?
\item What sorts of classification problems can be solved?
\item Can applying pruning rules to evolved functions, using domain
  knowledge, help the process?
\item How can we know when to stop a simulation? (Peak evaluation is
  usually reached soon)
 
\item What impact does the selection method have?
\end{enumerate}

\section{Programme Structure}
%\lstset{language=Rust}

Each evolved programme is a tree structure recursively defined 


\section{Selection}

There are several proposals for selection

\url{https://en.wikipedia.org/wiki/Stochastic_universal_sampling}

\url{https://en.wikipedia.org/wiki/Reward-based_selection}

\url{https://en.wikipedia.org/wiki/Tournament_selection}

\subsection{Roulette Wheel Selection}

\url{https://en.wikipedia.org/wiki/Fitness_proportionate_selection}

The usual method and the default method used herein.

\section{Fitness}

The general approach of
Koza\footnote{\url{http://www.genetic-programming.com/gpanimatedtutorial.html}}
is to view fitness as a scalar quantity.  A individual (programme) is
judged on its ability to correctly classify cases.

There is a limit to this view.  It may be well true that what is
required to identify some classes is poor at identifying othyers.
That is to say a individual is good at some specific cases whilst it
is poor generally.

This suggests that a more structured approach to fitness may be
appropriate where generalisation and spcificty are measured
differently.

\subsection{Specific and General Fitness: Consequences}

Using individuals developed with structured fitness implies that more
than one evolved individual should be involved in classification.
Individuals with high specific fitness can be used to identify cases
they are specialised for.

During evolution selection is more complex.  A way must be found to
select based on structured fitness.  Should they?

General selection shpould use general fitness

When selecting for crossover individuals based on special fitness they
should be matched with individuals specialised for the same class.

\subsubsection{Automatically Defined Functions}\footnote{``Evolving
  the Architecture of a Multi--Part Program in Genetic Programming
  Using Architecture-Altering Operations'' Koza 1995}

When a genetic programming system is extended to use automatically
defined functions (ADF) individuals with high specific fitness are
candidates to be used as ADFs available to evolving programmes.


  

\end{document}
